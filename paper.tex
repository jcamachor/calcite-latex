% This is "sig-alternate.tex" V2.1 April 2013
% This file should be compiled with V2.5 of "sig-alternate.cls" May 2012
%
% This example file demonstrates the use of the 'sig-alternate.cls'
% V2.5 LaTeX2e document class file. It is for those submitting
% articles to ACM Conference Proceedings WHO DO NOT WISH TO
% STRICTLY ADHERE TO THE SIGS (PUBS-BOARD-ENDORSED) STYLE.
% The 'sig-alternate.cls' file will produce a similar-looking,
% albeit, 'tighter' paper resulting in, invariably, fewer pages.
%
% ----------------------------------------------------------------------------------------------------------------
% This .tex file (and associated .cls V2.5) produces:
%       1) The Permission Statement
%       2) The Conference (location) Info information
%       3) The Copyright Line with ACM data
%       4) NO page numbers
%
% as against the acm_proc_article-sp.cls file which
% DOES NOT produce 1) thru' 3) above.
%
% Using 'sig-alternate.cls' you have control, however, from within
% the source .tex file, over both the CopyrightYear
% (defaulted to 200X) and the ACM Copyright Data
% (defaulted to X-XXXXX-XX-X/XX/XX).
% e.g.
% \CopyrightYear{2007} will cause 2007 to appear in the copyright line.
% \crdata{0-12345-67-8/90/12} will cause 0-12345-67-8/90/12 to appear in the copyright line.
%
% ---------------------------------------------------------------------------------------------------------------
% This .tex source is an example which *does* use
% the .bib file (from which the .bbl file % is produced).
% REMEMBER HOWEVER: After having produced the .bbl file,
% and prior to final submission, you *NEED* to 'insert'
% your .bbl file into your source .tex file so as to provide
% ONE 'self-contained' source file.
%
% ================= IF YOU HAVE QUESTIONS =======================
% Questions regarding the SIGS styles, SIGS policies and
% procedures, Conferences etc. should be sent to
% Adrienne Griscti (griscti@acm.org)
%
% Technical questions _only_ to
% Gerald Murray (murray@hq.acm.org)
% ===============================================================
%
% For tracking purposes - this is V2.0 - May 2012

\documentclass{sig-alternate-05-2015}


\begin{document}

% Copyright
\setcopyright{acmcopyright}
%\setcopyright{acmlicensed}
%\setcopyright{rightsretained}
%\setcopyright{usgov}
%\setcopyright{usgovmixed}
%\setcopyright{cagov}
%\setcopyright{cagovmixed}

\toappear{}

%% DOI
%\doi{10.475/123_4}
%
%% ISBN
%\isbn{123-4567-24-567/08/06}
%
%%Conference
%\conferenceinfo{PLDI '13}{June 16--19, 2013, Seattle, WA, USA}


\title{Apache Calcite: A Dynamic Data Management Framework}
%
% You need the command \numberofauthors to handle the 'placement
% and alignment' of the authors beneath the title.
%
% For aesthetic reasons, we recommend 'three authors at a time'
% i.e. three 'name/affiliation blocks' be placed beneath the title.
%
% NOTE: You are NOT restricted in how many 'rows' of
% "name/affiliations" may appear. We just ask that you restrict
% the number of 'columns' to three.
%
% Because of the available 'opening page real-estate'
% we ask you to refrain from putting more than six authors
% (two rows with three columns) beneath the article title.
% More than six makes the first-page appear very cluttered indeed.
%
% Use the \alignauthor commands to handle the names
% and affiliations for an 'aesthetic maximum' of six authors.
% Add names, affiliations, addresses for
% the seventh etc. author(s) as the argument for the
% \additionalauthors command.
% These 'additional authors' will be output/set for you
% without further effort on your part as the last section in
% the body of your article BEFORE References or any Appendices.

%\numberofauthors{8} %  in this sample file, there are a *total*
% of EIGHT authors. SIX appear on the 'first-page' (for formatting
% reasons) and the remaining two appear in the \additionalauthors section.
%
%\author{
% You can go ahead and credit any number of authors here,
% e.g. one 'row of three' or two rows (consisting of one row of three
% and a second row of one, two or three).
%
% The command \alignauthor (no curly braces needed) should
% precede each author name, affiliation/snail-mail address and
% e-mail address. Additionally, tag each line of
% affiliation/address with \affaddr, and tag the
% e-mail address with \email.
%
% 1st. author
%\alignauthor
%Ben Trovato\titlenote{Dr.~Trovato insisted his name be first.}\\
%       \affaddr{Institute for Clarity in Documentation}\\
%       \affaddr{1932 Wallamaloo Lane}\\
%       \affaddr{Wallamaloo, New Zealand}\\
%       \email{trovato@corporation.com}
%% 2nd. author
%\alignauthor
%G.K.M. Tobin\titlenote{The secretary disavows
%any knowledge of this author's actions.}\\
%       \affaddr{Institute for Clarity in Documentation}\\
%       \affaddr{P.O. Box 1212}\\
%       \affaddr{Dublin, Ohio 43017-6221}\\
%       \email{webmaster@marysville-ohio.com}
%% 3rd. author
%\alignauthor Lars Th{\o}rv{\"a}ld\titlenote{This author is the
%one who did all the really hard work.}\\
%       \affaddr{The Th{\o}rv{\"a}ld Group}\\
%       \affaddr{1 Th{\o}rv{\"a}ld Circle}\\
%       \affaddr{Hekla, Iceland}\\
%       \email{larst@affiliation.org}
%\and  % use '\and' if you need 'another row' of author names
%% 4th. author
%\alignauthor Lawrence P. Leipuner\\
%       \affaddr{Brookhaven Laboratories}\\
%       \affaddr{Brookhaven National Lab}\\
%       \affaddr{P.O. Box 5000}\\
%       \email{lleipuner@researchlabs.org}
%% 5th. author
%\alignauthor Sean Fogarty\\
%       \affaddr{NASA Ames Research Center}\\
%       \affaddr{Moffett Field}\\
%       \affaddr{California 94035}\\
%       \email{fogartys@amesres.org}
%% 6th. author
%\alignauthor Charles Palmer\\
%       \affaddr{Palmer Research Laboratories}\\
%       \affaddr{8600 Datapoint Drive}\\
%       \affaddr{San Antonio, Texas 78229}\\
%       \email{cpalmer@prl.com}
%}
% There's nothing stopping you putting the seventh, eighth, etc.
% author on the opening page (as the 'third row') but we ask,
% for aesthetic reasons that you place these 'additional authors'
% in the \additional authors block, viz.
%\additionalauthors{Additional authors: John Smith (The Th{\o}rv{\"a}ld Group,
%email: {\texttt{jsmith@affiliation.org}}) and Julius P.~Kumquat
%(The Kumquat Consortium, email: {\texttt{jpkumquat@consortium.net}}).}
%\date{30 July 1999}
% Just remember to make sure that the TOTAL number of authors
% is the number that will appear on the first page PLUS the
% number that will appear in the \additionalauthors section.

\maketitle
\begin{abstract}

\end{abstract}

% We no longer use \terms command
%\terms{Theory}

\keywords{keyword1; keyword2; keyword3}

\section{Introduction}
\label{sec:intro}

The challenge is not only the explosion of data size, but other factors like heterogeneity and velocity are decisive.
In recent years there has been a Cambrian explosion in the number of data processing systems that deal with these challenges.
While there might be differences in the processing capabilities of these systems, it is valuable to propose a common layer of abstraction or framework that can integrate and optimize with these systems.

Examples:
\begin{itemize}
	\item Integrating in-memory data (and a programming language model like LINQ) with a database-like approach (based on relational algebra).
	\item Hybrid queries (crossing multiple systems and memory).
	\item Materialized views, especially for OLAP applications, and especially dynamic (i.e. caching query results or intermediate results, and disabled when segments are flushed out of memory).
\end{itemize}

\subsection{Apache Calcite vision}
\label{subsec:vision}

The technical contributions of Calcite:

\begin{itemize}
	\item Open source friendly. Being written in Java, with Apache license, in ASF. So is the idea that code is divided into re-usable units (e.g. rules).
	\item More than one engine (Volcano, Hep) that run off the same rules. You can construct a multi-phase optimization process, using different engines, cost models and sets of traits (physical properties) in each phase. Rules can be re-used across phases (e.g. one phase might pull up projects, another might push them down).
	\item Streaming, and in particular integration of streaming and non-streaming data.
	\item The \textit{calling convention} trait, and plans that cross multiple engines. Calling convention is a physical property. And therefore it has a enforcer.
	\item Support for standard SQL, and also various dialects of SQL (e.g. Oracle SQL), case-sensitivity of identifiers, identifier quoting schemes.
	\item Avatica JDBC framework
	\item Metadata caching.
\end{itemize}



\subsection{Related work}
\label{subsec:related}

\todo{Add how Calcite uses concepts from Volcano and Cascade}

Today, many Big Data query processing systems that implement dialects of SQL such as Apache Hive, Apache Drill, Apache Phoenix and Apache Kylin use Calcite for query parsing and optimisation. Big Data query processing systems such as Presto, Spark SQL, HAWQ~\cite{chang2014hawq} and Impala~\cite{kornacker2015impala} use query planning systems specific to those systems. Then,  there are frameworks such as Orca~\cite{Soliman:2014:OMQ:2588555.2595637}, BigDAWG~\cite{duggan2015bigdawg}, Algebricks~\cite{borkar2015algebricks} and FORWARD~\cite{fu2011sql} with SQL++~\cite{ong2014sql++} support that can be used to implement query processing systems with capabilities such as support for polyglot storage backends, federated queries and semi-structured data models.

Orca is a modular query optimizer used in Pivotal's data management products such as  Greenplum and HAWQ.  Orca decouples the optimizer from query engine by implementing a framework for exchanging information between optimizer and query engine knows as \emph{Data eXchange Language}.  Orca also provides tools for verifying the correctness and performance of generated query plans. In contrast to Orca, Calcite can be used as a standalone query engine that federates multiple storage and processing backends. Calcite can also be used as an embeddable query optimizer. Calcite also supports multi-stage optimisations while Orca team was working on multi-stage optimisations at the time of writing of \cite{orca}. Algebricks provides a data model agnostic algebraic layer and compiler framework for Big Data query processing. High-level languages are compiled to Algebricks logical algebra and Algebricks takes care of generating a job target at specific processing backends such as Hyracks and Spark. Algebricks only supports rule-based optimisations whereas Calcite supports cost-based optimisations. FORWARD is federated query processor that implements a superset of SQL called SQL++. SQL++ has a semi-structured data model that extends both JSON and relational data model whereas Calcite supports semi-structured data models by representing them in relational data model during query planning. FORWARD decomposes federated queries written in SQL++ into subqueries that are compatible with underlying databases and execute them on underlying databases according to the query plan. The merging of data happens inside the FORWARD engine. BigDAWG is federated data storage and processing architecture that abstracts wide spectrum of data models including  relational, time-series and streaming. Unit of abstraction in BigDAWG is called \emph{island of information} and each island of information has a query language, data model and connects to one or more storage systems. Cross storage system querying is supported within the boundaries of a single island of information. In the context of BigDAWG architecture, Calcite can be considered as an island of information that supports SQL queries by implementing relational data model to abstract multiple storage systems. 




\section{Conclusion}
\label{sec:conclusion}

Some text.


%ACKNOWLEDGMENTS are optional
\section{Acknowledgments}
This section is optional; it is a location for you
to acknowledge grants, funding, editing assistance and
what have you.  In the present case, for example, the
authors would like to thank Gerald Murray of ACM for
his help in codifying this \textit{Author's Guide}
and the \textbf{.cls} and \textbf{.tex} files that it describes.

%
% The following two commands are all you need in the
% initial runs of your .tex file to
% produce the bibliography for the citations in your paper.
\bibliographystyle{abbrv}
\bibliography{ref}

%APPENDICES are optional
%\balancecolumns
%\appendix
%Appendix A
%\section{Headings in Appendices}

% That's all folks!
\end{document}
