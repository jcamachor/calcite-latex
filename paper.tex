% This is "sig-alternate.tex" V2.1 April 2013
% This file should be compiled with V2.5 of "sig-alternate.cls" May 2012
%
% This example file demonstrates the use of the 'sig-alternate.cls'
% V2.5 LaTeX2e document class file. It is for those submitting
% articles to ACM Conference Proceedings WHO DO NOT WISH TO
% STRICTLY ADHERE TO THE SIGS (PUBS-BOARD-ENDORSED) STYLE.
% The 'sig-alternate.cls' file will produce a similar-looking,
% albeit, 'tighter' paper resulting in, invariably, fewer pages.
%
% ----------------------------------------------------------------------------------------------------------------
% This .tex file (and associated .cls V2.5) produces:
%       1) The Permission Statement
%       2) The Conference (location) Info information
%       3) The Copyright Line with ACM data
%       4) NO page numbers
%
% as against the acm_proc_article-sp.cls file which
% DOES NOT produce 1) thru' 3) above.
%
% Using 'sig-alternate.cls' you have control, however, from within
% the source .tex file, over both the CopyrightYear
% (defaulted to 200X) and the ACM Copyright Data
% (defaulted to X-XXXXX-XX-X/XX/XX).
% e.g.
% \CopyrightYear{2007} will cause 2007 to appear in the copyright line.
% \crdata{0-12345-67-8/90/12} will cause 0-12345-67-8/90/12 to appear in the copyright line.
%
% ---------------------------------------------------------------------------------------------------------------
% This .tex source is an example which *does* use
% the .bib file (from which the .bbl file % is produced).
% REMEMBER HOWEVER: After having produced the .bbl file,
% and prior to final submission, you *NEED* to 'insert'
% your .bbl file into your source .tex file so as to provide
% ONE 'self-contained' source file.
%
% ================= IF YOU HAVE QUESTIONS =======================
% Questions regarding the SIGS styles, SIGS policies and
% procedures, Conferences etc. should be sent to
% Adrienne Griscti (griscti@acm.org)
%
% Technical questions _only_ to
% Gerald Murray (murray@hq.acm.org)
% ===============================================================
%
% For tracking purposes - this is V2.0 - May 2012

\documentclass{sig-alternate-05-2015}


\begin{document}

% Copyright
\setcopyright{acmcopyright}
%\setcopyright{acmlicensed}
%\setcopyright{rightsretained}
%\setcopyright{usgov}
%\setcopyright{usgovmixed}
%\setcopyright{cagov}
%\setcopyright{cagovmixed}

\toappear{}

%% DOI
%\doi{10.475/123_4}
%
%% ISBN
%\isbn{123-4567-24-567/08/06}
%
%%Conference
%\conferenceinfo{PLDI '13}{June 16--19, 2013, Seattle, WA, USA}


\title{Apache Calcite: A Dynamic Data Management Framework}
%
% You need the command \numberofauthors to handle the 'placement
% and alignment' of the authors beneath the title.
%
% For aesthetic reasons, we recommend 'three authors at a time'
% i.e. three 'name/affiliation blocks' be placed beneath the title.
%
% NOTE: You are NOT restricted in how many 'rows' of
% "name/affiliations" may appear. We just ask that you restrict
% the number of 'columns' to three.
%
% Because of the available 'opening page real-estate'
% we ask you to refrain from putting more than six authors
% (two rows with three columns) beneath the article title.
% More than six makes the first-page appear very cluttered indeed.
%
% Use the \alignauthor commands to handle the names
% and affiliations for an 'aesthetic maximum' of six authors.
% Add names, affiliations, addresses for
% the seventh etc. author(s) as the argument for the
% \additionalauthors command.
% These 'additional authors' will be output/set for you
% without further effort on your part as the last section in
% the body of your article BEFORE References or any Appendices.

%\numberofauthors{8} %  in this sample file, there are a *total*
% of EIGHT authors. SIX appear on the 'first-page' (for formatting
% reasons) and the remaining two appear in the \additionalauthors section.
%
%\author{
% You can go ahead and credit any number of authors here,
% e.g. one 'row of three' or two rows (consisting of one row of three
% and a second row of one, two or three).
%
% The command \alignauthor (no curly braces needed) should
% precede each author name, affiliation/snail-mail address and
% e-mail address. Additionally, tag each line of
% affiliation/address with \affaddr, and tag the
% e-mail address with \email.
%
% 1st. author
%\alignauthor
%Ben Trovato\titlenote{Dr.~Trovato insisted his name be first.}\\
%       \affaddr{Institute for Clarity in Documentation}\\
%       \affaddr{1932 Wallamaloo Lane}\\
%       \affaddr{Wallamaloo, New Zealand}\\
%       \email{trovato@corporation.com}
%% 2nd. author
%\alignauthor
%G.K.M. Tobin\titlenote{The secretary disavows
%any knowledge of this author's actions.}\\
%       \affaddr{Institute for Clarity in Documentation}\\
%       \affaddr{P.O. Box 1212}\\
%       \affaddr{Dublin, Ohio 43017-6221}\\
%       \email{webmaster@marysville-ohio.com}
%% 3rd. author
%\alignauthor Lars Th{\o}rv{\"a}ld\titlenote{This author is the
%one who did all the really hard work.}\\
%       \affaddr{The Th{\o}rv{\"a}ld Group}\\
%       \affaddr{1 Th{\o}rv{\"a}ld Circle}\\
%       \affaddr{Hekla, Iceland}\\
%       \email{larst@affiliation.org}
%\and  % use '\and' if you need 'another row' of author names
%% 4th. author
%\alignauthor Lawrence P. Leipuner\\
%       \affaddr{Brookhaven Laboratories}\\
%       \affaddr{Brookhaven National Lab}\\
%       \affaddr{P.O. Box 5000}\\
%       \email{lleipuner@researchlabs.org}
%% 5th. author
%\alignauthor Sean Fogarty\\
%       \affaddr{NASA Ames Research Center}\\
%       \affaddr{Moffett Field}\\
%       \affaddr{California 94035}\\
%       \email{fogartys@amesres.org}
%% 6th. author
%\alignauthor Charles Palmer\\
%       \affaddr{Palmer Research Laboratories}\\
%       \affaddr{8600 Datapoint Drive}\\
%       \affaddr{San Antonio, Texas 78229}\\
%       \email{cpalmer@prl.com}
%}
% There's nothing stopping you putting the seventh, eighth, etc.
% author on the opening page (as the 'third row') but we ask,
% for aesthetic reasons that you place these 'additional authors'
% in the \additional authors block, viz.
%\additionalauthors{Additional authors: John Smith (The Th{\o}rv{\"a}ld Group,
%email: {\texttt{jsmith@affiliation.org}}) and Julius P.~Kumquat
%(The Kumquat Consortium, email: {\texttt{jpkumquat@consortium.net}}).}
%\date{30 July 1999}
% Just remember to make sure that the TOTAL number of authors
% is the number that will appear on the first page PLUS the
% number that will appear in the \additionalauthors section.

\maketitle
\begin{abstract}

\end{abstract}

% We no longer use \terms command
%\terms{Theory}

\keywords{keyword1; keyword2; keyword3}

\section{Introduction}
\label{sec:intro}

The challenge is not only the explosion of data size, but other factors like heterogeneity and velocity are decisive.
In recent years there has been a Cambrian explosion in the number of data processing systems that deal with these challenges.
While there might be differences in the processing capabilities of these systems, it is valuable to propose a common layer of abstraction or framework that can integrate and optimize with these systems.

Examples:
\begin{itemize}
	\item Integrating in-memory data (and a programming language model like LINQ) with a database-like approach (based on relational algebra).
	\item Hybrid queries (crossing multiple systems and memory).
	\item Materialized views, especially for OLAP applications, and especially dynamic (i.e. caching query results or intermediate results, and disabled when segments are flushed out of memory).
\end{itemize}

\subsection{Apache Calcite vision}
\label{subsec:vision}

In this paper we describe Apache Calcite, a data management framework that aims at meeting the aforementioned goals. The dynamic nature of Calcite comes from its ability to adapt to the requirements of the underlying processing platforms depending on their specific needs. Calcite has enjoyed wide adoption since its inception, and more than a dozen systems widely used in industry already use it to power their query optimization logic~\footnote{\url{http://calcite.apache.org/docs/powered_by}}.

In the following, we enumerate some of the major features of Calcite that have contributed to the wide adoption of the framework.

\begin{itemize}
	\item\textbf{Open source friendly.} Calcite is an open-source framework, backed by the Apache Software Foundation (ASF)~\cite{asf:website}, which provides the means to collaboratively develop the project. Furthermore, the software is written in Java, making it easy to interoperate with many of the latest developed data processing systems~\cite{website:Drill,website:Flink,website:Hive,website:Kylin,website:Samza}, especially those in the Hadoop ecosystem.
	\item\textbf{Multiple data models.} Support for optimization of streaming and non-streaming data processing paradigms integrated within the same abstraction layer.
	\item\textbf{Flexible query optimizer.} Each component of the optimizer is pluggable and extensible, ranging from rules to cost models. In addition, Calcite includes support for multiple planning engines, making it easy to construct a multi-phase optimization process.
	\item\textbf{Cross-platform support.} The framework can optimize queries whose execution span across multiple processing systems. Even better, the decision is seamlessly integrated within the optimization logic by representing the system for each operator as a simple physical property.
	\item\textbf{Reliable and efficient implementation.} Calcite is reliable, as its wide adoption has led to exhaustive testing of the platform. Furthermore, its implementation includes techniques to improve the efficiency of the query optimization process, \eg\ smart caching of metadata information.
	\item\textbf{Extensions to work out-of-the-box.} Many systems do not provide their own query language, but rather prefer to rely on existing ones. For those, Calcite provides support for standard SQL, as well as various SQL dialects and extensions, \eg\ for expressing queries on streaming or nested data. In addition, Calcite includes a JDBC driver that can be used to easily integrate with the optimization framework.
\end{itemize}


\subsection{Related work}
\label{subsec:related}

Another ref~\cite{DBLP:conf/edbt/AgrawalCEKOPQ0Z16}.





\section{Conclusion}
\label{sec:conclusion}




%ACKNOWLEDGMENTS are optional
\section{Acknowledgments}
This section is optional; it is a location for you
to acknowledge grants, funding, editing assistance and
what have you.  In the present case, for example, the
authors would like to thank Gerald Murray of ACM for
his help in codifying this \textit{Author's Guide}
and the \textbf{.cls} and \textbf{.tex} files that it describes.

%
% The following two commands are all you need in the
% initial runs of your .tex file to
% produce the bibliography for the citations in your paper.
\bibliographystyle{abbrv}
\bibliography{ref}

%APPENDICES are optional
%\balancecolumns
%\appendix
%Appendix A
%\section{Headings in Appendices}

% That's all folks!
\end{document}
