% This is "sig-alternate.tex" V2.1 April 2013
% This file should be compiled with V2.5 of "sig-alternate.cls" May 2012
%
% This example file demonstrates the use of the 'sig-alternate.cls'
% V2.5 LaTeX2e document class file. It is for those submitting
% articles to ACM Conference Proceedings WHO DO NOT WISH TO
% STRICTLY ADHERE TO THE SIGS (PUBS-BOARD-ENDORSED) STYLE.
% The 'sig-alternate.cls' file will produce a similar-looking,
% albeit, 'tighter' paper resulting in, invariably, fewer pages.
%
% ----------------------------------------------------------------------------------------------------------------
% This .tex file (and associated .cls V2.5) produces:
%       1) The Permission Statement
%       2) The Conference (location) Info information
%       3) The Copyright Line with ACM data
%       4) NO page numbers
%
% as against the acm_proc_article-sp.cls file which
% DOES NOT produce 1) thru' 3) above.
%
% Using 'sig-alternate.cls' you have control, however, from within
% the source .tex file, over both the CopyrightYear
% (defaulted to 200X) and the ACM Copyright Data
% (defaulted to X-XXXXX-XX-X/XX/XX).
% e.g.
% \CopyrightYear{2007} will cause 2007 to appear in the copyright line.
% \crdata{0-12345-67-8/90/12} will cause 0-12345-67-8/90/12 to appear in the copyright line.
%
% ---------------------------------------------------------------------------------------------------------------
% This .tex source is an example which *does* use
% the .bib file (from which the .bbl file % is produced).
% REMEMBER HOWEVER: After having produced the .bbl file,
% and prior to final submission, you *NEED* to 'insert'
% your .bbl file into your source .tex file so as to provide
% ONE 'self-contained' source file.
%
% ================= IF YOU HAVE QUESTIONS =======================
% Questions regarding the SIGS styles, SIGS policies and
% procedures, Conferences etc. should be sent to
% Adrienne Griscti (griscti@acm.org)
%
% Technical questions _only_ to
% Gerald Murray (murray@hq.acm.org)
% ===============================================================
%
% For tracking purposes - this is V2.0 - May 2012

\documentclass{sig-alternate-05-2015}

\usepackage{graphicx}
\usepackage{hyperref}
\usepackage{multirow}
\usepackage[dvipsnames]{xcolor}
\usepackage{marvosym}
\usepackage{soul}
\usepackage{listings}
\usepackage{tikz}

% For draft only, remove before submission
\usepackage{todo}

% comments macros
\newcommand{\JH}[1]{\textcolor{ForestGreen}{\Pointinghand\textbf{JH}: #1}} %julian
\newcommand{\MM}[1]{\textcolor{red}{\Pointinghand\textbf{MM}: #1}} %michael
\newcommand{\MP}[1]{\textcolor{blue}{\Pointinghand\textbf{MP}: #1}} %milinda
\newcommand{\VS}[1]{\textcolor{magenta}{\Pointinghand\textbf{VS}: #1}} %vladimir
\newcommand{\JCR}[1]{\textcolor{orange}{\Pointinghand\textbf{JCR}: #1}} %jesus

%spacing
\newcommand{\va}{\vspace{-1.5mm}}
\newcommand{\vspacebfigure}{\vspace{-1.5mm}}   % vspace before caption of figure
\newcommand{\vspaceafigure}{\vspace{-2.5mm}}   % vspace after caption of figure

%shortcuts
\newcommand{\ie}{i.e.,~}
\newcommand{\eg}{e.g.,~}
\newcommand{\etc}{etc.}

%myparagraph
\newcommand{\myparagraph}[1]{\vspace{2mm}\noindent\textbf{#1}}

%listings
\definecolor{javared}{rgb}{0.6,0,0} % for strings
\definecolor{javagreen}{rgb}{0.25,0.5,0.35} % comments
\definecolor{javapurple}{rgb}{0.5,0,0.35} % keywords
\definecolor{javadocblue}{rgb}{0.25,0.35,0.75} % javadoc
 
\lstset{
basicstyle=\scriptsize\ttfamily,
keywordstyle=\color{javapurple}\bfseries,
stringstyle=\color{javared},
commentstyle=\color{javagreen},
morecomment=[s][\color{javadocblue}]{/**}{*/},
numbers=left,
numberstyle=\tiny\color{black},
stepnumber=0,
numbersep=10pt,
tabsize=4,
showspaces=false,
showstringspaces=false}

\makeatletter
\newenvironment{btHighlight}[1][]
{\begingroup\tikzset{bt@Highlight@par/.style={#1}}\begin{lrbox}{\@tempboxa}}
{\end{lrbox}\bt@HL@box[bt@Highlight@par]{\@tempboxa}\endgroup}

\newcommand\btHL[1][]{%
  \begin{btHighlight}[#1]\bgroup\aftergroup\bt@HL@endenv%
}
\def\bt@HL@endenv{%
  \end{btHighlight}%   
  \egroup
}
\newcommand{\bt@HL@box}[2][]{%
  \tikz[#1]{%
    \pgfpathrectangle{\pgfpoint{1pt}{0pt}}{\pgfpoint{\wd #2}{\ht #2}}%
    \pgfusepath{use as bounding box}%
    \node[anchor=base west, fill=orange!30,outer sep=0pt,inner xsep=1pt, inner ysep=0pt, minimum height=\ht\strutbox+1pt,#1]{\raisebox{1pt}{\strut}\strut\usebox{#2}};
  }%
}
\makeatother

\lstdefinestyle{STREAMINGSQL}{
    language={SQL},basicstyle=\scriptsize\ttfamily, 
    moredelim=**[is][\btHL]{`}{`},
    moredelim=**[is][{\btHL[fill=green!30,thin]}]{@}{@},
    keywordstyle=\color{javapurple}\bfseries,
		stringstyle=\color{javared},
		commentstyle=\color{javagreen},
		morecomment=[s][\color{javadocblue}]{/**}{*/},
		numbers=left,
		numberstyle=\tiny\color{black},
		stepnumber=0,
		numbersep=10pt,
		tabsize=4,
		showspaces=false,
		showstringspaces=false
}

%%
\begin{document}

% Copyright
\setcopyright{acmcopyright}
%\setcopyright{acmlicensed}
%\setcopyright{rightsretained}
%\setcopyright{usgov}
%\setcopyright{usgovmixed}
%\setcopyright{cagov}
%\setcopyright{cagovmixed}

\toappear{}

%% DOI
%\doi{10.475/123_4}
%
%% ISBN
%\isbn{123-4567-24-567/08/06}
%
%%Conference
%\conferenceinfo{PLDI '13}{June 16--19, 2013, Seattle, WA, USA}


\title{Apache Calcite: A Dynamic Data Management Framework}
%
% You need the command \numberofauthors to handle the 'placement
% and alignment' of the authors beneath the title.
%
% For aesthetic reasons, we recommend 'three authors at a time'
% i.e. three 'name/affiliation blocks' be placed beneath the title.
%
% NOTE: You are NOT restricted in how many 'rows' of
% "name/affiliations" may appear. We just ask that you restrict
% the number of 'columns' to three.
%
% Because of the available 'opening page real-estate'
% we ask you to refrain from putting more than six authors
% (two rows with three columns) beneath the article title.
% More than six makes the first-page appear very cluttered indeed.
%
% Use the \alignauthor commands to handle the names
% and affiliations for an 'aesthetic maximum' of six authors.
% Add names, affiliations, addresses for
% the seventh etc. author(s) as the argument for the
% \additionalauthors command.
% These 'additional authors' will be output/set for you
% without further effort on your part as the last section in
% the body of your article BEFORE References or any Appendices.

%\numberofauthors{8} %  in this sample file, there are a *total*
% of EIGHT authors. SIX appear on the 'first-page' (for formatting
% reasons) and the remaining two appear in the \additionalauthors section.
%
%\author{
% You can go ahead and credit any number of authors here,
% e.g. one 'row of three' or two rows (consisting of one row of three
% and a second row of one, two or three).
%
% The command \alignauthor (no curly braces needed) should
% precede each author name, affiliation/snail-mail address and
% e-mail address. Additionally, tag each line of
% affiliation/address with \affaddr, and tag the
% e-mail address with \email.
%
% 1st. author
%\alignauthor
%Ben Trovato\titlenote{Dr.~Trovato insisted his name be first.}\\
%       \affaddr{Institute for Clarity in Documentation}\\
%       \affaddr{1932 Wallamaloo Lane}\\
%       \affaddr{Wallamaloo, New Zealand}\\
%       \email{trovato@corporation.com}
%% 2nd. author
%\alignauthor
%G.K.M. Tobin\titlenote{The secretary disavows
%any knowledge of this author's actions.}\\
%       \affaddr{Institute for Clarity in Documentation}\\
%       \affaddr{P.O. Box 1212}\\
%       \affaddr{Dublin, Ohio 43017-6221}\\
%       \email{webmaster@marysville-ohio.com}
%% 3rd. author
%\alignauthor Lars Th{\o}rv{\"a}ld\titlenote{This author is the
%one who did all the really hard work.}\\
%       \affaddr{The Th{\o}rv{\"a}ld Group}\\
%       \affaddr{1 Th{\o}rv{\"a}ld Circle}\\
%       \affaddr{Hekla, Iceland}\\
%       \email{larst@affiliation.org}
%\and  % use '\and' if you need 'another row' of author names
%% 4th. author
%\alignauthor Lawrence P. Leipuner\\
%       \affaddr{Brookhaven Laboratories}\\
%       \affaddr{Brookhaven National Lab}\\
%       \affaddr{P.O. Box 5000}\\
%       \email{lleipuner@researchlabs.org}
%% 5th. author
%\alignauthor Sean Fogarty\\
%       \affaddr{NASA Ames Research Center}\\
%       \affaddr{Moffett Field}\\
%       \affaddr{California 94035}\\
%       \email{fogartys@amesres.org}
%% 6th. author
%\alignauthor Charles Palmer\\
%       \affaddr{Palmer Research Laboratories}\\
%       \affaddr{8600 Datapoint Drive}\\
%       \affaddr{San Antonio, Texas 78229}\\
%       \email{cpalmer@prl.com}
%}
% There's nothing stopping you putting the seventh, eighth, etc.
% author on the opening page (as the 'third row') but we ask,
% for aesthetic reasons that you place these 'additional authors'
% in the \additional authors block, viz.
%\additionalauthors{Additional authors: John Smith (The Th{\o}rv{\"a}ld Group,
%email: {\texttt{jsmith@affiliation.org}}) and Julius P.~Kumquat
%(The Kumquat Consortium, email: {\texttt{jpkumquat@consortium.net}}).}
%\date{30 July 1999}
% Just remember to make sure that the TOTAL number of authors
% is the number that will appear on the first page PLUS the
% number that will appear in the \additionalauthors section.

\maketitle
\begin{abstract}

\end{abstract}

% We no longer use \terms command
%\terms{Theory}

\keywords{keyword1; keyword2; keyword3}

\section{Introduction}
\label{sec:intro}

The challenge is not only the explosion of data size, but other factors like heterogeneity and velocity are decisive.
In recent years there has been a Cambrian explosion in the number of data processing systems that deal with these challenges.
While there might be differences in the processing capabilities of these systems, it is valuable to propose a common layer of abstraction or framework that can integrate and optimize with these systems.

Examples:
\begin{itemize}
	\item Integrating in-memory data (and a programming language model like LINQ) with a database-like approach (based on relational algebra).
	\item Hybrid queries (crossing multiple systems and memory).
	\item Materialized views, especially for OLAP applications, and especially dynamic (i.e. caching query results or intermediate results, and disabled when segments are flushed out of memory).
\end{itemize}

\subsection{Apache Calcite vision}
\label{subsec:vision}

In this paper we describe Apache Calcite, a data management framework that aims at meeting the aforementioned goals. The dynamic nature of Calcite comes from its ability to adapt to the requirements of the underlying processing platforms depending on their specific needs. Calcite has enjoyed wide adoption since its inception, and more than a dozen systems widely used in industry already use it to power their query optimization logic~\footnote{\url{http://calcite.apache.org/docs/powered_by}}.

In the following, we enumerate some of the major features of Calcite that have contributed to the wide adoption of the framework.

\begin{itemize}
	\item\textbf{Open source friendly.} Calcite is an open-source framework, backed by the Apache Software Foundation (ASF)~\cite{asf:website}, which provides the means to collaboratively develop the project. Furthermore, the software is written in Java, making it easy to interoperate with many of the latest developed data processing systems~\cite{website:Drill,website:Flink,website:Hive,website:Kylin,website:Samza}, especially those in the Hadoop ecosystem.
	\item\textbf{Multiple data models.} Support for optimization of streaming and non-streaming data processing paradigms integrated within the same abstraction layer.
	\item\textbf{Flexible query optimizer.} Each component of the optimizer is pluggable and extensible, ranging from rules to cost models. In addition, Calcite includes support for multiple planning engines, making it easy to construct a multi-phase optimization process.
	\item\textbf{Cross-platform support.} The framework can optimize queries whose execution span across multiple processing systems. Even better, the decision is seamlessly integrated within the optimization logic by representing the system for each operator as a simple physical property.
	\item\textbf{Reliable and efficient implementation.} Calcite is reliable, as its wide adoption has led to exhaustive testing of the platform. Furthermore, its implementation includes techniques to improve the efficiency of the query optimization process, \eg\ smart caching of metadata information.
	\item\textbf{Extensions to work out-of-the-box.} Many systems do not provide their own query language, but rather prefer to rely on existing ones. For those, Calcite provides support for standard SQL, as well as various SQL dialects and extensions, \eg\ for expressing queries on streaming or nested data. In addition, Calcite includes a JDBC driver that can be used to easily integrate with the optimization framework.
\end{itemize}


\subsection{Related work}
\label{subsec:related}

Another ref~\cite{DBLP:conf/edbt/AgrawalCEKOPQ0Z16}.



\section{Architecture}
\label{sec:archi}

Some text.\todo{Add figure with complete architecture}

\subsection{Data model and type system}
\label{subsec:dm-ts}
Calcite's data model consists of three main entities.

\begin{itemize}
  \item \textbf{Schema} - A schema groups tables, streams, views and materialisations into one logical entity. A schema adapter~\ref{subsec:adapters} can be used to expose particular kind of data (e.g. Cassandra keyspace) as tables within a schema to Calcite.
  \item \textbf{Table} - A table is a typed collection of records defined by a set of named, strongly typed columns. Records in a table have no defined ordering.
  \item \textbf{Stream} - A stream is a possibly indefinite sequence of temporally-defined typed records. Each record in a stream should contain at least one monotonic or quasi-monotonic column that defines the order of records temporarily.
\end{itemize}

\MP{Please feel free to edit above. I am still not sure what we should mention about type system.}


\subsection{Relational expressions}
\label{subsec:relexprs}

Relational algebra~\cite{DBLP:journals/cacm/Codd70} lies at the core of Calcite. In addition to the operators that can be used to express the most common data manipulation operations, such as \textit{filter}, \textit{project}, \textit{join} \etc , Calcite includes additional operators that meet different purposes, \eg being able to concisely represent complex operations, or recognize optimization opportunities more efficiently.

For instance, it has become common for OLAP, decision making, and streaming applications to use window definitions to express complex analytic functions such as moving average of a quantity over a time period or number or rows.  Thus, Calcite introduces a \textit{window} operator that encapsulates the window definition, \ie upper and lower bound, partitioning \etc, and the aggregate functions to execute on each window.

Nevertheless, it is recommended that users combine existing operators wherever possible, rather than defining new ones. The core relational algebra is expressive, and adding a new operator requires adding a planner rule for each combination of the new operator with existing operators.

%%
\myparagraph{Traits.} Calcite does not use different entities to represent logical and physical operators. Instead, it describes the physical properties associated with an operator using \textit{traits}. These traits help the optimizer evaluate the cost of different alternative plans. It is important to note that if an operator property is considered a trait, changing its value does not change the logical expression being evaluated, \ie the rows produced by the given operator will still be the same.

During optimization, Calcite will try to enforce certain traits on relational expressions, \eg the sort order of certain columns. Relational operators can implement a \textit{converter} interface that indicates how to convert the physical attribute of the expression from one value to another.

Calcite includes common traits that describe the physical properties of the data produced by a relational expression, such as \textit{ordering}, \textit{grouping}, and \textit{partitioning}. Similar to~\cite{DBLP:conf/icde/ZhouLC10}, the optimizer can reason about these properties and exploit them to find plans that avoid unnecessary operations.

In addition to these properties, one of the main features of Calcite is the \textit{calling convention} trait. Essentially, the trait represents the data processing system on which the expression will be be executed. Including the calling convention as a trait allows Calcite to meet its goal of optimizing transparently queries whose execution might span over different engines \ie the convention will be treated as any other physical property.

\todo{Add figure}For example, consider joining a \textit{Products} table held in MySQL to an \textit{Orders} table held in Splunk. Initially, the scan of \textit{Orders} takes place in \textit{splunk} convention and the scan of \textit{Products} is in \textit{jdbc-mysql} convention (the tables have to be scanned inside their respective engines), and the join is in logical convention (meaning that no implementation has been chosen). One possible implementation is to use Apache Spark as an external engine: the join is converted to \textit{spark} convention, and its inputs are converters from \textit{jdbc-mysql} and \textit{splunk} to \textit{spark} convention. But there is a more efficient implementation: exploiting the fact that Splunk can perform lookups into MySQL via ODBC, a planner rule pushes the join through the \textit{splunk}-to-\textit{spark} converter, and the join is now in \textit{splunk} convention, running inside the Splunk engine.



\subsection{Query optimizer}
\label{subsec:optimizer}

The query optimizer is the main component in the framework. Calcite optimizes queries by repeatedly applying planner rules to a relational expression. A cost model guides the process, and the planner engine tries to generate an alternative expression that has the same semantics as the original but a lower cost.

Every component in the optimizer is extensible. One can add his own relational operators, rewriting rules, cost model, statistics, and even planner engine.

%%
\myparagraph{Rewriting rules.} Calcite includes a set of planner rules to transform expression trees. In particular, a rules matches a given pattern in the tree and executes a transformation that preserves semantics of that expression. At the moment of this writing, Calcite rules account to more than 75. However, it is rather common that data processing systems that rely for optimization on Calcite include their own rules, \eg\ to explore rewritings especially beneficial in that system.

\todo{Add figure with example of complex rule}

%%
\myparagraph{Metadata providers.} 

%%
\myparagraph{Planner engines.} 

%%
\myparagraph{Materialized views}



\subsection{Schema adapters}
\label{subsec:adapters}

\JCR{Before getting into detail about the adapters and how to implement them, it would help the reader to mention shortly what an adapter is; another line about them might be added to the beginning of the Section 2}

As discussed in Section~\ref{subsec:relexprs}, Calcite uses a physical trait to identify relational algebra operators which correspond to a specific database backend.
These physical operators implement the access paths for the underlying tables in each adapter.
When a query is parsed and converted to a relational algebra expression, an operator is created for each table representing a scan of the data on that table.
This represents the minimal interface that an adapter must implement.
If an adapter implements the table scan operator, the Calcite optimizer is then able to use client side operators such as sorting, filtering, and joins to execute arbitrary SQL queries against these tables.

To implement a table scan, a table exposed by an adapter must first be able convert itself to a relational algebra node.
This node contains the necessary information the adapter will require to issue he scan to the adapter's backend database.
The node inherits calling convention of the adapter.
To extend the functionality provided by adapters, Calcite defines an \emph{enumerable} calling convention.
Relational algebra nodes with the enumerable calling convention simply operate over tuples which can accessed via an iterator interface.
This allows Calcite to implement operators which may not be available in each adapter's backend.
For example, the \texttt{EnmerableJoin} operator implements joins by collecting rows from the iterators of its child nodes and joining on the desired attributes.

However, for queries which only touch a small subset of the data in a table, it is highly inefficient for Calcite to process queries by enumerating all tuples in a table.
Fortunately, the same rule-based optimizer can be used to implement adapter-specific rules for optimization.
For example, suppose a query involves filtering and sorting on a table.
An adapter which can perform filtering on the backend can implement a rule which matches a \texttt{LogicalFilter} and converts it to the adapter's calling convention.
This rules converts the \texttt{LogicalFilter} into another \texttt{Filter} instance.
This new \texttt{Filter} node has an associated cost that allows Calcite to optimize queries across adapters.
This rule will then store the necessary information so that when the request is made to the backend, the adapter will return results which have already been filtered.
Note that if the adapter does not support sorting, this sorting can still be performed within Calcite.

The use of adapters is a powerful abstraction that allows not only the optimization of queries for a specific backend, but also across multiple backends.
Calcite is able to answer queries which involve tables across multiple backends by pushing down all possible logic to each backend and then performing joins and aggregations on the resulting data.
Implementing an adapter can be as simple as providing a table scan operator or can extend to many advanced optimizations.
Any expression represented in the relational algebra for the query can be pushed down to adapters with optimizer rules.


\subsection{Streaming}
\label{subsec:streaming}

Some text.


\section{Calcite in action}
\label{sec:action}

The best way to understand Calcite's capabilities is to review some of the systems that have been built using Calcite. Calcite's goal is to make it easier to build a DBMS, but as each DBMS is unique, each tends to use Calcite in its own particular way.

{\renewcommand{\tabcolsep}{2pt}
\begin{table*}[th]
\centering
{\small\begin{tabular}{|c|c|c|c|c|c|c|c|c|} \hline
\multirow{2}{*}{\textbf{Project}} & \multirow{2}{*}{\textbf{Frontend}} & \textbf{Avatica} & \textbf{SQL} & \textbf{SQL} & \textbf{Relational} & \multirow{2}{*}{\textbf{Adapters}} & \multirow{2}{*}{\textbf{Engine}} & \textbf{Use of} \\
 & & \textbf{JDBC} & \textbf{Parser} & \textbf{Validator} & \textbf{Algebra} & & & \textbf{Calcite} \\
\hline
\hline
Hive~\cite{website:Hive} & Hive SQL   & \checkmark &   &   &   &   & Hive              & Library  \\\hline
Kylin & SQL        & \checkmark & \checkmark & \checkmark &   &   & HBase, Enumerable & Library  \\\hline
Drill~\cite{website:Drill} & SQL        &   & \checkmark & \checkmark & \checkmark &   & Drill             & Library  \\\hline
Qubole Quark & SQL &   & \checkmark & \checkmark & \checkmark &   & Hive, Presto      & Embedded \\\hline
\multirow{2}{*}{Phoenix} & \multirow{2}{*}{SQL} & \multirow{2}{*}{\checkmark} & \multirow{2}{*}{\checkmark} & \multirow{2}{*}{\checkmark} & \multirow{2}{*}{\checkmark} &   & \multirow{2}{*}{HBase} & Local JDBC or \\
 & & & & & & & & query server \\\hline
Druid adapter & SQL or MDX & \checkmark & \checkmark & \checkmark & \checkmark & \checkmark & Druid, Enumerable & Local JDBC \\\hline
Lingual & SQL      &   & \checkmark & \checkmark & \checkmark &   & Cascading & Local JDBC \\\hline
\multirow{2}{*}{Calcite linq4j} & Java API &   &   & \multirow{2}{*}{\checkmark} &   &   & Enumerable & \multirow{2}{*}{API} \\
 & (similar to LINQ) &   &   &   &   &   & or any adapter & \\\hline
Samza SQL & Streaming SQL & \checkmark &   &   &   &   & Samza & Preprocessor \\\hline
Calcite JDBC & SQL & \checkmark & \checkmark & \checkmark & \checkmark & \checkmark & Enumerable, any adapter & Local JDBC \\\hline
\end{tabular}}
\caption{List of systems and its interactions with Calcite.\label{tab:systems}}\todo{Add refs for these systems}
\end{table*}
}

Table~\ref{tab:systems} lists some databases that use Calcite. For each, we consider: Which components did it use? How did it customize those components for its purposes? Did it use Calcite as a library, or did they plug itself into Calcite as an adapter? What query language(s) did it provide? And what engine(s) does it use to execute queries?
Apache Hive is a SQL interface to Apache Hadoop for ETL and interactive query, and uses Calcite for cost-based optimization. Hive uses its own SQL parser and validator (semantic analyzer) for its own dialect of SQL, translates to Calcite's logical algebra, which after optimization is translated to Hive's physical algebra and executed using Apache Tez or MapReduce.

Apache Kylin provides fast execution for interactive OLAP-style SQL queries. It rewrites JOIN-GROUP-BY queries to use materialized views stored in Apache HBase.

Qubole Quark provides fast execution for interactive SQL queries in Amazon Web Services. It dynamically builds materialized views, and uses 

Drill - flexible schema

SamzaSQL - streaming

Druid - MDX+SQL front ends

Cassandra - adapter - materialized views


Calcite is packaged in various ways. Hive embeds Calcite as a library. Drill, Kylin and Phoenix use Calcite's JDBC driver (based on Avatica), each making minor modifications to the driver to, for instance, change the connect string from ``jdbc:calcite:'' to ``jdbc:phoenix:''. Phoenix includes Avatica's query server, so that thin clients can connect via an RPC protocol based on protobuf.

\JH{The following terms should probably be converted into a reference on first use.
Protobuf
Apache Hive
Apache Kylin
Apache Drill
Apache Hadoop
Apache HBase
Cascading Lingual http://www.cascading.org/projects/lingual/ 
LINQ }


%\subsection{Interfaces}
\label{subsec:interfaces}

Some text.


\section{Next steps}
\label{sec:future}

Some text.

\section{Conclusion}
\label{sec:conclusion}




%ACKNOWLEDGMENTS are optional
\section{Acknowledgments}
This section is optional; it is a location for you
to acknowledge grants, funding, editing assistance and
what have you.  In the present case, for example, the
authors would like to thank Gerald Murray of ACM for
his help in codifying this \textit{Author's Guide}
and the \textbf{.cls} and \textbf{.tex} files that it describes.

%
% The following two commands are all you need in the
% initial runs of your .tex file to
% produce the bibliography for the citations in your paper.
\bibliographystyle{abbrv}
\bibliography{ref}

\todos

%APPENDICES are optional
%\balancecolumns
%\appendix
%Appendix A
%\section{Headings in Appendices}

% That's all folks!
\end{document}
