\subsection{Related work}
\label{subsec:related}

\todo{Add how Calcite uses concepts from Volcano and Cascade}

Today, many Big Data query processing systems that implement dialects of SQL such as Apache Hive, Apache Drill, Apache Phoenix and Apache Kylin use Calcite for query parsing and optimisation. Big Data query processing systems such as Presto, Spark SQL, HAWQ~\cite{chang2014hawq} and Impala~\cite{kornacker2015impala} use query planning systems specific to those systems. Then,  there are frameworks such as Orca~\cite{Soliman:2014:OMQ:2588555.2595637}, BigDAWG~\cite{duggan2015bigdawg}, Algebricks~\cite{borkar2015algebricks} and FORWARD~\cite{fu2011sql} with SQL++~\cite{ong2014sql++} support that can be used to implement query processing systems with capabilities such as support for polyglot storage backends, federated queries and semi-structured data models.

Orca is a modular query optimizer used in Pivotal's data management products such as  Greenplum and HAWQ.  Orca decouples the optimizer from query engine by implementing a framework for exchanging information between optimizer and query engine knows as \emph{Data eXchange Language}.  Orca also provides tools for verifying the correctness and performance of generated query plans. In contrast to Orca, Calcite can be used as a standalone query engine that federates multiple storage and processing backends. Calcite can also be used as an embeddable query optimizer. Calcite also supports multi-stage optimisations while Orca team was working on multi-stage optimisations at the time of writing of \cite{Soliman:2014:OMQ:2588555.2595637}. Algebricks provides a data model agnostic algebraic layer and compiler framework for Big Data query processing. High-level languages are compiled to Algebricks logical algebra and Algebricks takes care of generating a job target at specific processing backends such as Hyracks and Spark. Algebricks only supports rule-based optimisations whereas Calcite supports cost-based optimisations. FORWARD is federated query processor that implements a superset of SQL called SQL++. SQL++ has a semi-structured data model that extends both JSON and relational data model whereas Calcite supports semi-structured data models by representing them in relational data model during query planning. FORWARD decomposes federated queries written in SQL++ into subqueries that are compatible with underlying databases and execute them on underlying databases according to the query plan. The merging of data happens inside the FORWARD engine. BigDAWG is federated data storage and processing architecture that abstracts wide spectrum of data models including  relational, time-series and streaming. Unit of abstraction in BigDAWG is called \emph{island of information} and each island of information has a query language, data model and connects to one or more storage systems. Cross storage system querying is supported within the boundaries of a single island of information. In the context of BigDAWG architecture, Calcite can be considered as an island of information that supports SQL queries by implementing relational data model to abstract multiple storage systems. 
