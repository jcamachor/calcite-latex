\section{Architecture}
\label{sec:archi}

Calcite contains many of the pieces that comprise a typical database management system. However, it omits some key components, \eg storage of data, algorithms to process data, and a repository for storing metadata. This decision is intentional: this makes Calcite an excellent choice for mediating between applications having one or more data storage locations and using multiple data processing engines. It is also a perfect foundation for building your own data processing system.

\begin{figure}[t]
\centering
\includegraphics[width=0.94\columnwidth]{figures/architecture.png}
\vspacebfigure\caption{Apache Calcite architecture and interaction.\label{fig:arch}\JCR{Could we include the adapters in the figure?}}\vspaceafigure
\end{figure}

Figure~\ref{fig:arch} outlines the main components of Calcite's architecture. In the Figure, the dashed lines represent possible external interactions with the framework.

First of all, Calcite contains a query parser and validator that can translate a SQL query to a tree of relational operators. As we mentioned earlier, Calcite does not contain a \textit{storage layer}. Hence, it provides a mechanism to define table schemas and views using JSON files, so it can be used as a stand-alone system.

The tree of relational operators is the internal representation on which Calcite's optimizer works. The optimization engine consists mainly of three components which guide the process: rules, metadata providers, and planner engines. We talk about these components in more detail in Section~\ref{subsec:optimizer}.

Once the query has been optimized, Calcite can translate the relational expression back to SQL. Note that this allows Calcite to work as a stand-alone system on top of any data application with a SQL interface.

However, Calcite architecture is not only tailored towards this fashion of interacting with the framework. In fact, it is far more common that data processing systems choose to use their own parser for their own query language. In that case, Calcite also allows to easily build the operator tree directly instantiating the relational operators. Alternately, one can use the built-in \textit{relational expressions builder} interface. For instance, assume that we want to express the following SQL query using the expressions builder:

\begin{lstlisting}[language=SQL]
SELECT deptno, count(*) AS c, sum(sal) AS s
FROM emp
GROUP BY deptno;
\end{lstlisting}

The equivalent expression looks as follows:

\begin{lstlisting}[language=Java]
final RelNode node = builder
  .scan("EMP")
  .aggregate(builder.groupKey("DEPTNO"),
      builder.count(false, "C"),
      builder.sum(false, "S", builder.field("SAL")))
  .build();
\end{lstlisting}

Observe that the interface exposes the main constructs for building relational expressions. After the optimization phase is finished, the application can retrieve the optimized relational expression.

\subsection{Data model and type system}
\label{subsec:dm-ts}
Calcite's data model consists of three main entities.

\begin{itemize}
  \item \textbf{Schema} - A schema groups tables, streams, views and materialisations into one logical entity. A schema adapter~\ref{subsec:adapters} can be used to expose particular kind of data (e.g. Cassandra keyspace) as tables within a schema to Calcite.
  \item \textbf{Table} - A table is a typed collection of records defined by a set of named, strongly typed columns. Records in a table have no defined ordering.
  \item \textbf{Stream} - A stream is a possibly indefinite sequence of temporally-defined typed records. Each record in a stream should contain at least one monotonic or quasi-monotonic column that defines the order of records temporarily.
\end{itemize}

\MP{Please feel free to edit above. I am still not sure what we should mention about type system.}


\subsection{Relational expressions}
\label{subsec:relexprs}

Some text.


\subsection{Query optimizer}
\label{subsec:optimizer}

The query optimizer is the main component in the framework. Calcite optimizes queries by repeatedly applying planner rules to a relational expression. A cost model guides the process, and the planner engine tries to generate an alternative expression that has the same semantics as the original but a lower cost.

Every component in the optimizer is extensible. One can add his own relational operators, rewriting rules, cost model, statistics, and even planner engine.

%%
\myparagraph{Rewriting rules.} Calcite includes a set of planner rules to transform expression trees. In particular, a rules matches a given pattern in the tree and executes a transformation that preserves semantics of that expression. At the moment of this writing, Calcite rules account to more than 75. However, it is rather common that data processing systems that rely for optimization on Calcite include their own rules, \eg\ to explore rewritings especially beneficial in that system.

\todo{Add figure with example of complex rule}

%%
\myparagraph{Metadata providers.} 

%%
\myparagraph{Planner engines.} 

%%
\myparagraph{Materialized views}



\subsection{Schema adapters}
\label{subsec:adapters}

Some text.


\subsection{Streaming}
\label{subsec:streaming}

Some text.


