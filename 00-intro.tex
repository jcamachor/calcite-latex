\section{Introduction}
\label{sec:intro}

In the last decade, we have experienced an explosion in the volume, velocity, and heterogeneity of the data produced by individuals and companies through Internet, mobile devices, wearables, \etc\ In parallel, there has been an explosion in the number of data systems used to process this data and extract actionable value from it, usually targeting a rather specific use-case. About a decade ago, Stonebraker and \c{C}etintemel~\cite{DBLP:conf/icde/StonebrakerC05} already predicted that ``one size fits all" was an idea that was coming to an end: that vision seems today more relevant than ever.

As organizations continue investing heavily on creating new data processing systems tailored towards their specific needs, it seems to be valuable to propose a common layer of abstraction or framework that can integrate easily with these systems and aid their queries optimization. The motivation to use such a framework is many-fold. Among others:

\begin{itemize}
	\item Reducing end-to-end time to create data processing applications with an optimizer at the core. Multiple recent data processing systems that enjoy wide adoption did not include a query optimizer until later in their development~\cite{DBLP:conf/sigmod/ArmbrustXLHLBMK15,DBLP:conf/sigmod/HuaiCGHHOPYL014}. In contrast, integrating with such a framework, developers can focus at first on the processing capabilities of their systems, without sacrificing the benefits that complex query optimization can bring to performance. Furthermore, the optimization logic must not be developed multiple times, but rather consolidated and reused among systems.
	\item Hiding the complexity of the optimization process. If required, a developer will be able to run query optimization with the toolbox provided in the framework by default. Simultaneously, one should be able to customize the optimization process by being able to develop and plug extensions, \eg custom planners, operators, rules, or providers for metadata information.
	\item Integrating query languages with different expressive power~\cite{DBLP:journals/cacm/Hyde10,DBLP:conf/sigmod/MeijerBB06}, possibly tailored towards different processing abstractions, within the same formal model on which query optimizations can be applied.%based on algebraic expressions 
	\item Enabling cross-platform optimization by exposing a common interface to all the systems. This is an important feature, as the number of data processing systems, as well as the need for them to interact with each other, will continue increasing. Hence, the optimizer needs to reason globally to improve the performance of data applications, \eg\ making decisions across different backends about materialized view selection.
\end{itemize}

The idea of having a common framework does not come without challenges. In particular, the framework needs to be extensible and flexible enough to accommodate all the different type of systems integrating with it.


%%
\subsection{Apache Calcite vision}
\label{subsec:vision}

In this paper we describe Apache Calcite, a data management framework that aims at meeting the aforementioned goals. The dynamic nature of Calcite comes from its ability to adapt to the requirements of the underlying processing platforms depending on their specific needs. Calcite has enjoyed wide adoption since its inception, and more than a dozen systems widely used in industry already use it to power their query optimization logic~\footnote{\url{http://calcite.apache.org/docs/powered_by}}.

In the following, we enumerate some of the major features of Calcite that have contributed to the wide adoption of the framework.

\begin{itemize}
	\item\textbf{Open source friendly.} Calcite is an open-source framework, backed by the Apache Software Foundation (ASF)~\cite{asf:website}, which provides the means to collaboratively develop the project. Furthermore, the software is written in Java, making it easy to interoperate with many of the latest developed data processing systems~\cite{website:Drill,website:Flink,website:Hive,website:Kylin,website:Samza}, especially those in the Hadoop ecosystem.
	\item\textbf{Multiple data models.} Support for optimization of streaming and non-streaming data processing paradigms integrated within the same abstraction layer.
	\item\textbf{Flexible query optimizer.} Each component of the optimizer is pluggable and extensible, ranging from rules to cost models. In addition, Calcite includes support for multiple planning engines, making it easy to construct a multi-phase optimization process.
	\item\textbf{Cross-platform support.} The framework can optimize queries whose execution span across multiple processing systems. Even better, the decision is seamlessly integrated within the optimization logic by representing the system for each operator as a simple physical property.
	\item\textbf{Reliable and efficient implementation.} Calcite is reliable, as its wide adoption has led to exhaustive testing of the platform. Furthermore, its implementation includes techniques to improve the efficiency of the query optimization process, \eg\ smart caching of metadata information.
	\item\textbf{Extensions to work out-of-the-box.} Many systems do not provide their own query language, but rather prefer to rely on existing ones. For those, Calcite provides support for standard SQL, as well as various SQL dialects and extensions, \eg\ for expressing queries on streaming or nested data. In addition, Calcite includes a JDBC driver that can be used to easily integrate with the optimization framework.
\end{itemize}



%%
\myparagraph{Outline.} The remainder of this paper is organized
as follows. Section~\ref{sec:archi} presents Calcite's architecture and its main components. In turn, Section~\ref{sec:action} overviews the data processing systems that are already using Calcite. Section~\ref{sec:future} discusses possible future extensions for the framework. Finally, Section~\ref{sec:related} discusses related work, and then we conclude.
