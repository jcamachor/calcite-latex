\section{Introduction}
\label{sec:intro}

In the last decade, we have experienced an explosion in the volume, velocity, and heterogeneity of the data produced by individuals and companies through Internet, mobile devices, wearables, \etc\ In parallel, there has been an explosion in the number of data systems used to process this data and extract actionable value from it, usually targeting a rather specific use-case. About a decade ago, Stonebraker and \c{C}etintemel~\cite{DBLP:conf/icde/StonebrakerC05} already predicted that ``one size fits all" was an idea that was coming to an end: that vision seems today more relevant than ever.

As organizations continue investing heavily on creating new data processing systems tailored towards their specific needs, it seems to be valuable to propose a common layer of abstraction or framework that can integrate easily with these systems and aid their optimization. The reason to use such a framework would be many-fold. Among others:

\begin{itemize}
	\item Reducing end-to-end time to create data processing applications with an optimizer at the core. Multiple recent data processing systems that enjoy wide adoption did not include a query optimizer until later in their development~\cite{DBLP:conf/sigmod/ArmbrustXLHLBMK15,DBLP:conf/sigmod/HuaiCGHHOPYL014}. In contrast, integrating with such a framework, developers can focus at first on the processing capabilities of their systems, without sacrificing the benefits that complex query optimization can bring to performance. Furthermore, the optimization logic must not be developed multiple times, but rather consolidated and reused among systems.
	\item Hiding the complexity of the optimization process. If required, a developer will be able to run query optimization with the toolbox provided in the framework by default. Simultaneously, one should be able to customize the optimization process by being able to develop and plug extensions, \eg custom planners, operators, rules, or providers for metadata information.
	\item Integrating query languages with different expressive power~\cite{DBLP:journals/cacm/Hyde10,DBLP:conf/sigmod/MeijerBB06}, possibly tailored towards different processing abstractions, within the same formal model based on algebraic expressions on which query optimizations can be applied.
	\item Enabling cross-platform optimization by exposing a common interface to all the systems. This is an important feature, as the number of data processing systems as well as the need for them to interact with each other will continue increasing. Hence, the optimizer needs to reason globally to improve the performance of data applications, \eg\ making decisions across different backends about materialized view selection.
\end{itemize}

The idea of having a common framework does not come without challenges. In particular, the framework needs to be extensible and flexible enough to accommodate all the different type of systems integrating with it.


%%
\subsection{Apache Calcite vision}
\label{subsec:vision}

The technical contributions of Calcite:

\begin{itemize}
	\item Open source friendly. Being written in Java, with Apache license, in ASF. So is the idea that code is divided into re-usable units (e.g. rules).
	\item More than one engine (Volcano, Hep) that run off the same rules. You can construct a multi-phase optimization process, using different engines, cost models and sets of traits (physical properties) in each phase. Rules can be re-used across phases (e.g. one phase might pull up projects, another might push them down).
	\item Streaming, and in particular integration of streaming and non-streaming data.
	\item The \textit{calling convention} trait, and plans that cross multiple engines. Calling convention is a physical property. And therefore it has a enforcer.
	\item Support for standard SQL, and also various dialects of SQL (e.g. Oracle SQL), case-sensitivity of identifiers, identifier quoting schemes.
	\item Avatica JDBC framework
	\item Metadata caching.
\end{itemize}




%%
\myparagraph{Outline.} The remainder of this paper is organized
as follows. Section~\ref{sec:archi} presents Calcite's architecture and its main components. In turn, Section~\ref{sec:action} overviews the data processing systems that are already using Calcite. Section~\ref{sec:future} discusses possible future extensions for the framework. Finally, Section~\ref{sec:related} discusses related work, and then we conclude.
